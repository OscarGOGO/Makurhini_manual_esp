% Options for packages loaded elsewhere
\PassOptionsToPackage{unicode}{hyperref}
\PassOptionsToPackage{hyphens}{url}
%
\documentclass[
]{book}
\usepackage{amsmath,amssymb}
\usepackage{iftex}
\ifPDFTeX
  \usepackage[T1]{fontenc}
  \usepackage[utf8]{inputenc}
  \usepackage{textcomp} % provide euro and other symbols
\else % if luatex or xetex
  \usepackage{unicode-math} % this also loads fontspec
  \defaultfontfeatures{Scale=MatchLowercase}
  \defaultfontfeatures[\rmfamily]{Ligatures=TeX,Scale=1}
\fi
\usepackage{lmodern}
\ifPDFTeX\else
  % xetex/luatex font selection
\fi
% Use upquote if available, for straight quotes in verbatim environments
\IfFileExists{upquote.sty}{\usepackage{upquote}}{}
\IfFileExists{microtype.sty}{% use microtype if available
  \usepackage[]{microtype}
  \UseMicrotypeSet[protrusion]{basicmath} % disable protrusion for tt fonts
}{}
\makeatletter
\@ifundefined{KOMAClassName}{% if non-KOMA class
  \IfFileExists{parskip.sty}{%
    \usepackage{parskip}
  }{% else
    \setlength{\parindent}{0pt}
    \setlength{\parskip}{6pt plus 2pt minus 1pt}}
}{% if KOMA class
  \KOMAoptions{parskip=half}}
\makeatother
\usepackage{xcolor}
\usepackage{color}
\usepackage{fancyvrb}
\newcommand{\VerbBar}{|}
\newcommand{\VERB}{\Verb[commandchars=\\\{\}]}
\DefineVerbatimEnvironment{Highlighting}{Verbatim}{commandchars=\\\{\}}
% Add ',fontsize=\small' for more characters per line
\usepackage{framed}
\definecolor{shadecolor}{RGB}{248,248,248}
\newenvironment{Shaded}{\begin{snugshade}}{\end{snugshade}}
\newcommand{\AlertTok}[1]{\textcolor[rgb]{0.94,0.16,0.16}{#1}}
\newcommand{\AnnotationTok}[1]{\textcolor[rgb]{0.56,0.35,0.01}{\textbf{\textit{#1}}}}
\newcommand{\AttributeTok}[1]{\textcolor[rgb]{0.13,0.29,0.53}{#1}}
\newcommand{\BaseNTok}[1]{\textcolor[rgb]{0.00,0.00,0.81}{#1}}
\newcommand{\BuiltInTok}[1]{#1}
\newcommand{\CharTok}[1]{\textcolor[rgb]{0.31,0.60,0.02}{#1}}
\newcommand{\CommentTok}[1]{\textcolor[rgb]{0.56,0.35,0.01}{\textit{#1}}}
\newcommand{\CommentVarTok}[1]{\textcolor[rgb]{0.56,0.35,0.01}{\textbf{\textit{#1}}}}
\newcommand{\ConstantTok}[1]{\textcolor[rgb]{0.56,0.35,0.01}{#1}}
\newcommand{\ControlFlowTok}[1]{\textcolor[rgb]{0.13,0.29,0.53}{\textbf{#1}}}
\newcommand{\DataTypeTok}[1]{\textcolor[rgb]{0.13,0.29,0.53}{#1}}
\newcommand{\DecValTok}[1]{\textcolor[rgb]{0.00,0.00,0.81}{#1}}
\newcommand{\DocumentationTok}[1]{\textcolor[rgb]{0.56,0.35,0.01}{\textbf{\textit{#1}}}}
\newcommand{\ErrorTok}[1]{\textcolor[rgb]{0.64,0.00,0.00}{\textbf{#1}}}
\newcommand{\ExtensionTok}[1]{#1}
\newcommand{\FloatTok}[1]{\textcolor[rgb]{0.00,0.00,0.81}{#1}}
\newcommand{\FunctionTok}[1]{\textcolor[rgb]{0.13,0.29,0.53}{\textbf{#1}}}
\newcommand{\ImportTok}[1]{#1}
\newcommand{\InformationTok}[1]{\textcolor[rgb]{0.56,0.35,0.01}{\textbf{\textit{#1}}}}
\newcommand{\KeywordTok}[1]{\textcolor[rgb]{0.13,0.29,0.53}{\textbf{#1}}}
\newcommand{\NormalTok}[1]{#1}
\newcommand{\OperatorTok}[1]{\textcolor[rgb]{0.81,0.36,0.00}{\textbf{#1}}}
\newcommand{\OtherTok}[1]{\textcolor[rgb]{0.56,0.35,0.01}{#1}}
\newcommand{\PreprocessorTok}[1]{\textcolor[rgb]{0.56,0.35,0.01}{\textit{#1}}}
\newcommand{\RegionMarkerTok}[1]{#1}
\newcommand{\SpecialCharTok}[1]{\textcolor[rgb]{0.81,0.36,0.00}{\textbf{#1}}}
\newcommand{\SpecialStringTok}[1]{\textcolor[rgb]{0.31,0.60,0.02}{#1}}
\newcommand{\StringTok}[1]{\textcolor[rgb]{0.31,0.60,0.02}{#1}}
\newcommand{\VariableTok}[1]{\textcolor[rgb]{0.00,0.00,0.00}{#1}}
\newcommand{\VerbatimStringTok}[1]{\textcolor[rgb]{0.31,0.60,0.02}{#1}}
\newcommand{\WarningTok}[1]{\textcolor[rgb]{0.56,0.35,0.01}{\textbf{\textit{#1}}}}
\usepackage{longtable,booktabs,array}
\usepackage{calc} % for calculating minipage widths
% Correct order of tables after \paragraph or \subparagraph
\usepackage{etoolbox}
\makeatletter
\patchcmd\longtable{\par}{\if@noskipsec\mbox{}\fi\par}{}{}
\makeatother
% Allow footnotes in longtable head/foot
\IfFileExists{footnotehyper.sty}{\usepackage{footnotehyper}}{\usepackage{footnote}}
\makesavenoteenv{longtable}
\usepackage{graphicx}
\makeatletter
\newsavebox\pandoc@box
\newcommand*\pandocbounded[1]{% scales image to fit in text height/width
  \sbox\pandoc@box{#1}%
  \Gscale@div\@tempa{\textheight}{\dimexpr\ht\pandoc@box+\dp\pandoc@box\relax}%
  \Gscale@div\@tempb{\linewidth}{\wd\pandoc@box}%
  \ifdim\@tempb\p@<\@tempa\p@\let\@tempa\@tempb\fi% select the smaller of both
  \ifdim\@tempa\p@<\p@\scalebox{\@tempa}{\usebox\pandoc@box}%
  \else\usebox{\pandoc@box}%
  \fi%
}
% Set default figure placement to htbp
\def\fps@figure{htbp}
\makeatother
\setlength{\emergencystretch}{3em} % prevent overfull lines
\providecommand{\tightlist}{%
  \setlength{\itemsep}{0pt}\setlength{\parskip}{0pt}}
\setcounter{secnumdepth}{5}
\usepackage{booktabs}
\usepackage[]{natbib}
\bibliographystyle{apalike}
\usepackage{bookmark}
\IfFileExists{xurl.sty}{\usepackage{xurl}}{} % add URL line breaks if available
\urlstyle{same}
\hypersetup{
  pdftitle={Manual de uso de Makurhini},
  pdfauthor={Oscar Godínez Gómez},
  hidelinks,
  pdfcreator={LaTeX via pandoc}}

\title{Manual de uso de Makurhini}
\author{Oscar Godínez Gómez}
\date{2025-07-16}

\begin{document}
\maketitle

{
\setcounter{tocdepth}{1}
\tableofcontents
}
\chapter*{Makurhini}\label{makurhini}
\addcontentsline{toc}{chapter}{Makurhini}

\includegraphics[width=2.08333in,height=\textheight,keepaspectratio]{LOGO_MAKHURINI.png}

En este manual se documenta parte de las funciones del paquete Makurhini

\chapter{Instalación del paquete Makurhini}\label{instalaciuxf3n-del-paquete-makurhini}

\section{Instalación estandar}\label{instalaciuxf3n-estandar}

\begin{itemize}
\tightlist
\item
  Depende de R (\textgreater{} 4.0.0), igraph (\textgreater= 1.2.6)
\end{itemize}

\begin{Shaded}
\begin{Highlighting}[]
\FunctionTok{install.packages}\NormalTok{(}\StringTok{"igraph"}\NormalTok{)}
\end{Highlighting}
\end{Shaded}

\begin{itemize}
\item
  Se recomienda pre-instalar Rtools:
  \url{https://cran.r-project.org/bin/windows/Rtools/}
\item
  Se recomienda pre-instalar los paquetes devtools y remotes
\end{itemize}

\begin{Shaded}
\begin{Highlighting}[]
\FunctionTok{install.packages}\NormalTok{(}\FunctionTok{c}\NormalTok{(}\StringTok{"devtools"}\NormalTok{, }\StringTok{"remotes"}\NormalTok{))}
\end{Highlighting}
\end{Shaded}

\begin{Shaded}
\begin{Highlighting}[]
\FunctionTok{library}\NormalTok{(devtools) }
\FunctionTok{library}\NormalTok{(remotes)}
\FunctionTok{install\_github}\NormalTok{(}\StringTok{"connectscape/Makurhini"}\NormalTok{, }\AttributeTok{dependencies =} \ConstantTok{TRUE}\NormalTok{, }\AttributeTok{upgrade =} \StringTok{"never"}\NormalTok{)}
\end{Highlighting}
\end{Shaded}

En caso de que no aparezca en la lista de paquetes, cierre la sesión de
R y vuelva a abrirla.

\textbf{Si se produce el siguiente error durante la instalación:}

\begin{Shaded}
\begin{Highlighting}[]
\NormalTok{Using github PAT}
\NormalTok{from envvar GITHUB\_PAT Error}\SpecialCharTok{:}\NormalTok{ Failed to install }\StringTok{\textquotesingle{}unknown package\textquotesingle{}}\NormalTok{ from}
\NormalTok{GitHub}\SpecialCharTok{:}\NormalTok{ HTTP error }\FloatTok{401.}\NormalTok{ Bad credentials}
\end{Highlighting}
\end{Shaded}

\textbf{Entonces intenta lo siguiente:}

\begin{Shaded}
\begin{Highlighting}[]
\FunctionTok{Sys.getenv}\NormalTok{(}\StringTok{"GITHUB\_PAT"}\NormalTok{) }
\FunctionTok{Sys.unsetenv}\NormalTok{(}\StringTok{"GITHUB\_PAT"}\NormalTok{)}
\end{Highlighting}
\end{Shaded}

\section{Instalar en Linux}\label{instalar-en-linux}

Makurhini en Linux Para instalar Makurhini en linux considere los
siguientes pasos:

Utilice la línea de comandos de Linux para instalar el paquete de la
unidad:

\begin{Shaded}
\begin{Highlighting}[]
\NormalTok{sudo apt}\SpecialCharTok{{-}}\NormalTok{get install }\SpecialCharTok{{-}}\NormalTok{y libudunits2}\SpecialCharTok{{-}}\NormalTok{dev}
\end{Highlighting}
\end{Shaded}

Utilice la línea de comandos de Linux para instalar gdal:

\begin{Shaded}
\begin{Highlighting}[]
\NormalTok{sudo apt install libgdal}\SpecialCharTok{{-}}\NormalTok{dev}
\end{Highlighting}
\end{Shaded}

Utilice la línea de comandos de Linux para instalar libfontconfig y
libharfbuzz:

\begin{Shaded}
\begin{Highlighting}[]
\NormalTok{sudo apt install libfontconfig1}\SpecialCharTok{{-}}\NormalTok{dev}

\NormalTok{sudo apt install libharfbuzz}\SpecialCharTok{{-}}\NormalTok{dev libfribidi}\SpecialCharTok{{-}}\NormalTok{dev}
\end{Highlighting}
\end{Shaded}

Ahora puede instalar los paquetes devtools y remotes, y los paquetes
terra, raster y sf directamente en su R o RStudio.

\begin{Shaded}
\begin{Highlighting}[]
\FunctionTok{install.packages}\NormalTok{(}\FunctionTok{c}\NormalTok{(}\StringTok{\textquotesingle{}remotes\textquotesingle{}}\NormalTok{, }\StringTok{\textquotesingle{}devtools\textquotesingle{}}\NormalTok{, }\StringTok{\textquotesingle{}terra\textquotesingle{}}\NormalTok{, }\StringTok{\textquotesingle{}raster\textquotesingle{}}\NormalTok{, }\StringTok{\textquotesingle{}sf\textquotesingle{}}\NormalTok{))}
\end{Highlighting}
\end{Shaded}

Utiliza la línea de comandos de Linux para instalar igraph:

\begin{Shaded}
\begin{Highlighting}[]
\NormalTok{sudo apt}\SpecialCharTok{{-}}\NormalTok{get install libnlopt}\SpecialCharTok{{-}}\NormalTok{dev}

\NormalTok{sudo apt}\SpecialCharTok{{-}}\NormalTok{get install r}\SpecialCharTok{{-}}\NormalTok{cran}\SpecialCharTok{{-}}\NormalTok{igraph}
\end{Highlighting}
\end{Shaded}

Ahora puede instalar los paquetes gdistance, graph4lg y ggpubr
directamente en su R o RStudio.

\begin{Shaded}
\begin{Highlighting}[]
\FunctionTok{install.packages}\NormalTok{(}\FunctionTok{c}\NormalTok{(}\StringTok{\textquotesingle{}gdistance\textquotesingle{}}\NormalTok{, }\StringTok{\textquotesingle{}graph4lg\textquotesingle{}}\NormalTok{, }\StringTok{\textquotesingle{}ggpubr\textquotesingle{}}\NormalTok{))}
\end{Highlighting}
\end{Shaded}

Ahora puedes instalar Makurhini directamente en tu R o RStudio.

\begin{Shaded}
\begin{Highlighting}[]
\FunctionTok{library}\NormalTok{(devtools) }
\FunctionTok{library}\NormalTok{(remotes) }
\FunctionTok{install\_github}\NormalTok{(}\StringTok{"connectscape/Makurhini"}\NormalTok{, }\AttributeTok{dependencies =} \ConstantTok{TRUE}\NormalTok{, }\AttributeTok{upgrade =} \StringTok{"never"}\NormalTok{) }
\end{Highlighting}
\end{Shaded}

Tenga en cuenta que la instalación de Makurhini en Linux depende de su
versión del sistema operativo y de que consiga instalar los paquetes de
los que depende Makurhini.

  \bibliography{book.bib,packages.bib}

\end{document}
